\documentclass[]{standalone}
\usepackage{tikz}
\usetikzlibrary{shapes,arrows,calc,positioning}
\usepackage{amsmath} % for dfrac

% definition of basic block
\tikzset{
    block/.style = {draw, rectangle,
        minimum height=1.2cm,
        minimum width=2cm},
    input/.style = {coordinate,node distance=1cm},
    output/.style = {coordinate,node distance=1cm},
    sum/.style = {draw, circle, node distance=1cm},
}

% definition of saturation block
\tikzset{% from https://tex.stackexchange.com/questions/161075/saturation-block
  saturation block/.style={%
    draw, 
    path picture={
      % Get the width and height of the path picture node
      \pgfpointdiff{\pgfpointanchor{path picture bounding box}{north east}}%
        {\pgfpointanchor{path picture bounding box}{south west}}
      \pgfgetlastxy\x\y
      % Scale the x and y vectors so that the range
      % -1 to 1 is slightly shorter than the size of the node
      \tikzset{x=\x*.4, y=\y*.4}
      %
      % Draw annotation
      \draw (-1,0) -- (1,0) (0,-1) -- (0,1); 
      \draw (-1,-.7) -- (-.6,-.7) -- (.6,.7) -- (1,.7);
    }
  }
}
\tikzset{% from https://tex.stackexchange.com/questions/161075/saturation-block
  deadband block/.style={%
    draw, 
    path picture={
      % Get the width and height of the path picture node
      \pgfpointdiff{\pgfpointanchor{path picture bounding box}{north east}}%
        {\pgfpointanchor{path picture bounding box}{south west}}
      \pgfgetlastxy\x\y
      % Scale the x and y vectors so that the range
      % -1 to 1 is slightly shorter than the size of the node
      \tikzset{x=\x*.4, y=\y*.4}
      %
      % Draw annotation
      \draw (-1,0) -- (1,0) (0,-1) -- (0,1);  % axis
      \draw (-1,1) -- (-.3,.3) -- (-.3,0) -- (.3,0) -- (.3,-.3) -- (1,-1);
	  %\draw (-.3,.3) -- (.3,-.3) ;
    }
  }
}

\begin{document}
	\begin{tikzpicture}[auto, node distance=1cm,>=triangle 45]
		% Starting input (wref)
		\node [input, name=input, label=$\omega_{\text{ref}_{PU}}$] {};
		% sum 1
		\node [sum, right=of input] (sum1) {$\Sigma$};
		% delay block w
		\node [block, right=of sum1,label=17:*] (delay1) {$\dfrac{e^{-\$D_1}}{1+\$T_1}$};
		
		% delta w node and label
		\coordinate [right=of delay1]  (deltaw) {};
		\node [above=-.8em of deltaw,label={$\Delta\omega_{PU}$}]  (deltaWlabel){};
		
		% Hz Deadband
		\node [deadband block, right=of deltaw, minimum size=3.5em,label=17:*] (deadband) {};
		% delta w gain blocks
		\node [block, right=of deadband] (gain) {$\dfrac{M_{Base}}{R}$};
		\node [block, below=of gain] (Dt) {$M_{Base} Dt$};
		% Pref sum
		\node [sum, right=of gain] (sumP) {$\Sigma$};
		% delay block Pref
		\node [block, above=of sumP,label=17:*] (delay2) {$\dfrac{e^{-\$D_2}}{1+\$T_2}$};
		
		% limiter and labels
		\node [saturation block, right= of sumP , minimum size=3.5em, label=$MW_{cap}$] (mwcap){};
		\node [below=2em of mwcap, label=$0.0$](mwcapLOW){};
		% Valve state block
		\node [block, right=of mwcap] (state1) {$\dfrac{1}{1+\$T_s}$};
		
		% turbine state
		\node [block, right=of state1] (state2) {$\dfrac{1+\$T_3}{1+\$T_c}$};
		% Gov state
		\node [block, right=of state2] (state3) {$\dfrac{1+\$T_4}{1+\$T_5}$};
		% damping sum
		\node [sum, right= of state3] (sum2) {$\Sigma$};
		% Pm out
		\node [output, right=of sum2, label=$P_M$] (output) {};
		% w and pref in
		\node [input, name=omega, above= of sum1,label={[label distance=.1cm]0:$\omega_{PU}$} ] {};
		\node [input, name=Pref, above= of delay2,label={[label distance=.1cm]0:$P_{\text{ref}}$} ] {};

		% connecting lines
		\draw [draw,->] (input) -- node[pos=0.8] {$+$} (sum1); % straight connecting line
		\draw [->] (omega) -- node[pos=0.8] {$-$} (sum1);
		\draw [->] (Pref) --  (delay2);
		\draw [->] (delay2) -- node[pos=0.8] {$+$} (sumP) ;	
		\draw [->] (sum1) -- (delay1) ;		
		\draw [->] (delay1) -- (deadband) ;
		\draw [->] (deadband) -- (gain) ;
		\draw [->] (gain) -- (sumP) ;
		\draw [->] (sumP) -- (mwcap) ;
		\draw [->] (mwcap) -- (state1) ;
		\draw [->] (state1) -- (state2) ;
		\draw [->] (state2) -- (state3) ;
		\draw [->] (state3) -- node[pos=0.8] {$+$} (sum2);
		\draw [->] (deltaw) |-  (Dt); % line goes down and across
		\draw [->] (Dt) -|  node[pos=0.9] {$-$} (sum2); % line goes across then down
		\draw [->] (sum2) -- (output);
	\end{tikzpicture} 
\end{document}