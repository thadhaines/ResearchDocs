\documentclass[12pt]{article}
\usepackage[english]{babel}
\usepackage[utf8]{inputenc}

%% Pointer to 'default' preamble, other reusable files
% pacakages and definitions

\usepackage{geometry}
\geometry{
	letterpaper, 
	portrait, 
	top=.75in,
	left=.8in,
	right=.75in,
	bottom=.5in		} 	% Page Margins
	
%% additional packages for nice things
\usepackage{amsmath} 	% for most math
\usepackage{commath} 	% for abs
\usepackage{lastpage}	% for page count
\usepackage{amssymb} 	% for therefore
\usepackage{graphicx} 	% for image handling
\usepackage{wrapfig} 	% wrap figures
\usepackage[none]{hyphenat} % for no hyphenations
\usepackage{array} 		% for >{} column characterisctis
\usepackage{physics} 	% for easier derivative \dv....
\usepackage{tikz} 		% for graphic@!
\usepackage{circuitikz} % for circuits!
\usetikzlibrary{arrows.meta} % for loads
\usepackage[thicklines]{cancel}	% for cancels
\usepackage{xcolor}		% for color cancels
\usepackage[per-mode=fraction]{siunitx} % for si units and num
\sisetup{group-separator = {,}, group-minimum-digits = 3} % additional si unit table functionality

\usepackage{fancyhdr} 	% for header
\usepackage{comment}	% for ability to comment out large sections
\usepackage{multicol}	% for multiple columns using multicols
\usepackage[framed,numbered]{matlab-prettifier} % matlab sytle listing
\usepackage{marvosym} 	% for boltsymbol lightning
\usepackage{pdflscape} 	% for various landscape pages in portrait docs.
%\usepackage{float}
\usepackage{fancyvrb}	% for Verbatim (a tab respecting verbatim)
\usepackage{enumitem}	% for [resume] functionality of enumerate
\usepackage{spreadtab} 	% for using formulas in tables}
\usepackage{numprint}	% for number format in spread tab
\usepackage{subcaption} % for subfigures with captions
\usepackage[normalem]{ulem} % for strike through sout

% for row colors in tables....
\usepackage{color, colortbl}
\definecolor{G1}{gray}{0.9}
\definecolor{G2}{rgb}{1,0.88,1}%{gray}{0.6}
\definecolor{G3}{rgb}{0.88,1,1}

% For table formatting
\usepackage{booktabs}
\renewcommand{\arraystretch}{1.2}
\usepackage{floatrow}
\floatsetup[table]{capposition=top} % put table captions on top of tables

% Caption formating footnotesize ~ 10 pt in a 12 pt document
\usepackage[font={small}]{caption}

%% package config 
\sisetup{output-exponent-marker=\ensuremath{\mathrm{E}}} % for engineer E
\renewcommand{\CancelColor}{\color{red}}	% for color cancels
\lstset{aboveskip=2pt,belowskip=2pt} % for more compact table
%\arraycolsep=1.4pt\def
\setlength{\parindent}{0cm} % Remove indentation from paragraphs
\setlength{\columnsep}{0.5cm}
\lstset{
	style      = Matlab-editor,
	basicstyle = \ttfamily\footnotesize, % if you want to use Courier - not really used?
}
\renewcommand*{\pd}[3][]{\ensuremath{\dfrac{\partial^{#1} #2}{\partial #3}}} % for larger pd fracs
\renewcommand{\real}[1]{\mathbb{R}\left\{ #1 \right\}}	% for REAL symbol
\newcommand{\imag}[1]{\mathbb{I}\left\{ #1 \right\}}	% for IMAG symbol
\definecolor{m}{rgb}{1,0,1}	% for MATLAB matching magenta
	
%% custom macros
\newcommand\numberthis{\addtocounter{equation}{1}\tag{\theequation}} % for simple \numberthis command

\newcommand{\equal}{=} % so circuitikz can have an = in the labels
\newcolumntype{L}[1]{>{\raggedright\let\newline\\\arraybackslash\hspace{0pt}}m{#1}}
\newcolumntype{C}[1]{>{\centering\let\newline\\\arraybackslash\hspace{0pt}}m{#1}}
\newcolumntype{R}[1]{>{\raggedleft\let\newline\\\arraybackslash\hspace{0pt}}m{#1}}

%% Header
\pagestyle{fancy} % for header stuffs
\fancyhf{}
% spacing
\headheight 29 pt
\headsep 6 pt
%%% custom commands for nicer units
\newcommand{\mw}{\ensuremath{\text{ MW}}}
\newcommand{\hz}{\ensuremath{\text{ Hz}}}
\newcommand{\pu}{\ensuremath{\text{ Pu}}}
\newcommand{\sbase}{\ensuremath{\text{S}_{\text{Base}}}}
\newcommand{\fbase}{\ensuremath{f_{\text{Base}}}}
\newcommand{\mbase}[1]{\ensuremath{\text{M}_{\text{Base}_{#1}}}}
\newcommand{\hsys}{\ensuremath{\text{ H}_{\text{sys}}}}


%% Header
\rhead{Thad Haines \\ Page \thepage\ of \pageref{LastPage}}
\chead{Research Update \\ Week of January 6th, 2020}
\lhead{Research \\ }

%\usepackage{graphicx}
%\graphicspath{ {figures/} }
%\newcommand{\caseName}{ }

\begin{document}
\begin{multicols}{2}
\raggedright
	\paragraph{Thesis Schedule:}
	\begin{enumerate}
\itemsep0em 
		\item Draft thesis to Donnelly and Southergill Week of\textbf{ Feb 10}.
		\item Revised thesis to Committee week of \\ \textbf{Mar 9} (pre-spring break).
		\item Thesis Defense week of \textbf{April 13}.
		\item Final thesis and docs to Southergill week of \textbf{April 20}.
\item Other tasks:
\subitem Complete other graduation forms
\subitem Book room for defense
\subitem Get EIT references
\end{enumerate}

	\paragraph{Recent Progress:}
	\begin{enumerate}
\itemsep0em 

\item Register for graduation
\item Branch Flow calculation correction
\subitem New calculations: \\
\begin{align}
	I &= \dfrac{ V_S e^{j\delta_S}-V_R e^{j\delta_R}}{ (R+jX)\sqrt{3}}\\
	P &= V_S \sqrt{3} \abs{I} \cos(\delta_S -\angle{I}) \\
	Q &= V_S \sqrt{3} \abs{I} \sin(\delta_S -\angle{I}) 
\end{align}
\subitem Old calculations: \\
\begin{align}
P &= \dfrac{V_R V_S}{X} \sin(\delta_S - \delta_R)\\
Q &= \dfrac{V_R}{X} \left(V_S\cos(\delta_S - \delta_R)-V_R\right)\\
I &= \dfrac{\abs{P + jQ}}{V_R\sqrt{3}}
\end{align}

		
		
		\item GitHub updated:\\
		\verb|https://github.com/thadhaines/|


\vfill\null
\columnbreak
	
	\end{enumerate}
\paragraph{Current Tasks:}
	\begin{enumerate}
		\itemsep0em 
		\item work on feed forward governor design
		\item work on feed forward gov scenario
\item Create daily load cycle agent to read EIA data (hourly forecast and demand values)
		\item Solidify test cases for engineering problem
		\item Update Code flowchart and finalize code% to aid in further development.
		\item Thesis work 
\end{enumerate}
\paragraph{Proposed MiniWECC test cases:} \ \\
duration: 4-6 hours
\begin{itemize}
	\itemsep0em 
\item system noise 
\item wind generation ramps 
\item daily load cycle (during peak/valley transition)
\end{itemize}

Control varaitions:\\
Normal gov deadband and large gov deadband \\
Fast (seconds) and slow (minutes) AGC
\\ Three cases: 
\begin{itemize}
	\itemsep0em 
\item normal gov, Slow AGC
\item normal gov, Fast AGC
\item large gov, Fast AGC
\end{itemize}

Experimental Measures:
\begin{itemize}
\itemsep 0em
\item Valve movement
\item NERC mandate adherence
\end{itemize}

		%\item Keep Goals and Requests in mind.
		
		%\subitem A FlowtabrDAO exists that can find flow between busses. A way to initialize bus connections between areas has yet to be devised.




	\paragraph{Current Questions:}
	\begin{enumerate}
\itemsep0em 
	\item Progress on case data?
	\item VAR calculation - Real power and AMPS match, Reactive power off (see reverse)
	\end{enumerate}
	



\begin{comment}
\paragraph{Future Tasks:} %(Little to No Progress since last time / Things coming down the pipe)
	\begin{enumerate}
		
		\item Add import mirror / bypass mirror init sequence option to prevent repeated mirror creations.

		\item Bring wind into simulation \\ (ramp ungoverned generators?)


		
	\end{enumerate}
\paragraph{Future Work: (not by me)}
\begin{itemize}
\item Find best/correct way to trip gens in PSLF from python.
\item Account for different types of loads better. (exponential load model) % read from dyd
\item Work to incorporate Matt's \emph{Suggested Use Cases} into simulation.
		\begin{itemize}
		\item Add Shunt Group Agent
		\item Work to Define Definite Time Controller user input
		\end{itemize} 


		\item Investigate ULTC action.

		\item Create an agent for every object: \\ ULTC, SVD, Transformer, \ldots

		\item Move away from reliance on GE
		
\end{itemize}

\paragraph{Matt Requests:}
\begin{enumerate}
		\item Enable multiple dyd files to overwrite / replace previously defined agents/parameters
		\item Allow for variable time steps.
\end{enumerate}
\end{comment}

\vfill\null
\end{multicols}


\pagebreak

\newcommand{\caseName}{SixMachineRamp1}
\begin{landscape}

\includegraphics[width=.33\linewidth,height=.33\textheight]{figures/\caseName Pbr1}%
\includegraphics[width=.33\linewidth,height=.33\textheight]{figures/\caseName Pbr2}%
\includegraphics[width=.33\linewidth,height=.33\textheight]{figures/\caseName Pbr3}

\includegraphics[width=.33\linewidth,height=.33\textheight]{figures/\caseName Qbr1}%
\includegraphics[width=.33\linewidth,height=.33\textheight]{figures/\caseName Qbr2}%
\includegraphics[width=.33\linewidth,height=.33\textheight]{figures/\caseName Qbr3}

\includegraphics[width=.33\linewidth,height=.33\textheight]{figures/\caseName Amp1}%
\includegraphics[width=.33\linewidth,height=.33\textheight]{figures/\caseName Amp2}%
\includegraphics[width=.33\linewidth,height=.33\textheight]{figures/\caseName Amp3}

\end{landscape}

%\paragraph{'Soft Goals':}
%	\begin{enumerate}
%	\item Write Thesis 2020
%	\end{enumerate}
		

\end{document}