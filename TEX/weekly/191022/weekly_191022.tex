\documentclass[12pt]{article}
\usepackage[english]{babel}
\usepackage[utf8]{inputenc}

%% Pointer to 'default' preamble, other reusable files
\input{../../thad_preamble.tex}
\input{../../thad_cmds.tex}

%% Header
\rhead{Thad Haines \\ Page \thepage\ of \pageref{LastPage}}
\chead{Talking Points \\ Week of October 21th, 2019}
\lhead{Research \\ }

%\usepackage{graphicx}
%\graphicspath{ {figures/} }
%\newcommand{\caseName}{ }

\begin{document}
\begin{multicols}{2}
\raggedright
	\paragraph{Recent Progress:}
	\begin{enumerate}
\itemsep0em 
		\item Noise Agent Created

		\item Deadband Experimental Results

	%	\item More \verb|matplotlib| plot functions created.

		\item GitHub updated:\\
		\verb|https://github.com/thadhaines/|
		
	\end{enumerate}
\paragraph{Current Tasks:}
	\begin{enumerate}
		\itemsep0em 
		\item Paper for IEEE PES
		%\item Generic Governor testing
		\item Continue to refine BA ACE actions.
		%\item Update Code flowchart% to aid in further development.
		\item Thesis work 
		%\item Keep Goals and Requests in mind.
		
		%\subitem A FlowtabrDAO exists that can find flow between busses. A way to initialize bus connections between areas has yet to be devised.

	\end{enumerate}

	\paragraph{Current Questions:}
	\begin{enumerate}
\itemsep0em 
	\item Realistic AGC results and/or tuning?
	\item Typical deadbands of AGC? 
	\item IEEE Paper outline or title?\\Long-Term Effect of Governor Deadband on Valve Travel
	\end{enumerate}
	
\paragraph{Deadband Explained} \ \\
\includegraphics[width=\linewidth]{tgov1DB}\\
\includegraphics[width=\linewidth]{dbAction2}

\paragraph{MiniWECC AGC Settings}

\begin{itemize}
\itemsep0em 
\item 15 second ACG Action Time
\item PI filtered ACE
\item 15 Second Windowed IACE included
\item Step Deadband at 36 mHz
\item N-L Droop from 16-36 mHz
\end{itemize}
\vfill\null
\columnbreak
\paragraph{MiniWECC Noise Results} \ \\
System Loading (0.05\% Noise Added):\\
\includegraphics[width=\linewidth]{miniWECCnoiseStepDBPload}
Step DB:\\
\includegraphics[width=\linewidth]{miniWECCnoiseStepDBFreq}
\includegraphics[width=\linewidth]{miniWECCnoiseStepDBValveTravel01}
Non-linear Droop DB:\\
\includegraphics[width=\linewidth]{miniWECCnoiseNLDBFreq}
\includegraphics[width=\linewidth]{miniWECCnoiseNLDBValveTravel01}
\includegraphics[width=.9\linewidth]{miniWECCnoiseRes01}

\vfill\null
\end{multicols}
%\pagebreak
\paragraph{AGC Block Diagram} \ \\

\includegraphics[width=\linewidth]{AGC-TLB}\\
Optional Filters:\\
\begin{center}
\includegraphics[width=.8\linewidth]{filterAgent}
\end{center}

\paragraph{Controller Results (no noise)} miniWECC 1500 MW generation loss at t=2\ \\
No Deadband:\\
\includegraphics[width=\linewidth]{noDB}
{Step Deadband:\\
\includegraphics[width=\linewidth]{stepDB}
Non-Linear Droop:\\
\includegraphics[width=\linewidth]{NLdroopDB}
\begin{comment}
% Common 2nd col removed 

\paragraph{Future Tasks:} %(Little to No Progress since last time / Things coming down the pipe)
	\begin{enumerate}
		
		\item Add import mirror / bypass mirror init sequence option to prevent repeated mirror creations.

		\item Bring wind into simulation \\ (ramp ungoverned generators?)

		\item Find best/correct way to trip gens in PSLF from python.

		\item Investigate line current data.
		
	\end{enumerate}
\paragraph{Future Work: (not by me)}
\begin{itemize}
\item Account for different types of loads. (exponential load model) % read from dyd
\item Work to incorporate Matt's \emph{Suggested Use Cases} into simulation.
		\begin{itemize}
		\item Add Shunt Group Agent
		\item Work to Define Definite Time Controller user input
		\end{itemize} 


		\item Investigate ULTC action.

		\item Create an agent for every object: \\ ULTC, SVD, Transformer, \ldots

		\item Get away from reliance on GE
		
\end{itemize}

\paragraph{Matt Requests:}
\begin{enumerate}
		\item Enable multiple dyd files to overwrite / replace previously defined agents/parameters
		\item Allow for variable time steps.
\end{enumerate}

	\end{enumerate}


\pagebreak
\end{comment}


%\paragraph{'Soft Goals':}
%	\begin{enumerate}
%	\item Write Thesis 2020
%	\end{enumerate}
		

\begin{comment}

\end{comment}

\end{document}