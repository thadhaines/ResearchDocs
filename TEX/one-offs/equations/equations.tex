% Equations used in project
% Started 07/22/219

\documentclass[12pt]{article}
\usepackage[english]{babel}
\usepackage[utf8]{inputenc}

%% Pointer to 'default' preamble, other reusable files
% pacakages and definitions

\usepackage{geometry}
\geometry{
	letterpaper, 
	portrait, 
	top=.75in,
	left=.8in,
	right=.75in,
	bottom=.5in		} 	% Page Margins
	
%% additional packages for nice things
\usepackage{amsmath} 	% for most math
\usepackage{commath} 	% for abs
\usepackage{lastpage}	% for page count
\usepackage{amssymb} 	% for therefore
\usepackage{graphicx} 	% for image handling
\usepackage{wrapfig} 	% wrap figures
\usepackage[none]{hyphenat} % for no hyphenations
\usepackage{array} 		% for >{} column characterisctis
\usepackage{physics} 	% for easier derivative \dv....
\usepackage{tikz} 		% for graphic@!
\usepackage{circuitikz} % for circuits!
\usetikzlibrary{arrows.meta} % for loads
\usepackage[thicklines]{cancel}	% for cancels
\usepackage{xcolor}		% for color cancels
\usepackage[per-mode=fraction]{siunitx} % for si units and num
\sisetup{group-separator = {,}, group-minimum-digits = 3} % additional si unit table functionality

\usepackage{fancyhdr} 	% for header
\usepackage{comment}	% for ability to comment out large sections
\usepackage{multicol}	% for multiple columns using multicols
\usepackage[framed,numbered]{matlab-prettifier} % matlab sytle listing
\usepackage{marvosym} 	% for boltsymbol lightning
\usepackage{pdflscape} 	% for various landscape pages in portrait docs.
%\usepackage{float}
\usepackage{fancyvrb}	% for Verbatim (a tab respecting verbatim)
\usepackage{enumitem}	% for [resume] functionality of enumerate
\usepackage{spreadtab} 	% for using formulas in tables}
\usepackage{numprint}	% for number format in spread tab
\usepackage{subcaption} % for subfigures with captions
\usepackage[normalem]{ulem} % for strike through sout

% for row colors in tables....
\usepackage{color, colortbl}
\definecolor{G1}{gray}{0.9}
\definecolor{G2}{rgb}{1,0.88,1}%{gray}{0.6}
\definecolor{G3}{rgb}{0.88,1,1}

% For table formatting
\usepackage{booktabs}
\renewcommand{\arraystretch}{1.2}
\usepackage{floatrow}
\floatsetup[table]{capposition=top} % put table captions on top of tables

% Caption formating footnotesize ~ 10 pt in a 12 pt document
\usepackage[font={small}]{caption}

%% package config 
\sisetup{output-exponent-marker=\ensuremath{\mathrm{E}}} % for engineer E
\renewcommand{\CancelColor}{\color{red}}	% for color cancels
\lstset{aboveskip=2pt,belowskip=2pt} % for more compact table
%\arraycolsep=1.4pt\def
\setlength{\parindent}{0cm} % Remove indentation from paragraphs
\setlength{\columnsep}{0.5cm}
\lstset{
	style      = Matlab-editor,
	basicstyle = \ttfamily\footnotesize, % if you want to use Courier - not really used?
}
\renewcommand*{\pd}[3][]{\ensuremath{\dfrac{\partial^{#1} #2}{\partial #3}}} % for larger pd fracs
\renewcommand{\real}[1]{\mathbb{R}\left\{ #1 \right\}}	% for REAL symbol
\newcommand{\imag}[1]{\mathbb{I}\left\{ #1 \right\}}	% for IMAG symbol
\definecolor{m}{rgb}{1,0,1}	% for MATLAB matching magenta
	
%% custom macros
\newcommand\numberthis{\addtocounter{equation}{1}\tag{\theequation}} % for simple \numberthis command

\newcommand{\equal}{=} % so circuitikz can have an = in the labels
\newcolumntype{L}[1]{>{\raggedright\let\newline\\\arraybackslash\hspace{0pt}}m{#1}}
\newcolumntype{C}[1]{>{\centering\let\newline\\\arraybackslash\hspace{0pt}}m{#1}}
\newcolumntype{R}[1]{>{\raggedleft\let\newline\\\arraybackslash\hspace{0pt}}m{#1}}

%% Header
\pagestyle{fancy} % for header stuffs
\fancyhf{}
% spacing
\headheight 29 pt
\headsep 6 pt
%%% custom commands for nicer units
\newcommand{\mw}{\ensuremath{\text{ MW}}}
\newcommand{\hz}{\ensuremath{\text{ Hz}}}
\newcommand{\pu}{\ensuremath{\text{ Pu}}}
\newcommand{\sbase}{\ensuremath{\text{S}_{\text{Base}}}}
\newcommand{\fbase}{\ensuremath{f_{\text{Base}}}}
\newcommand{\mbase}[1]{\ensuremath{\text{M}_{\text{Base}_{#1}}}}
\newcommand{\hsys}{\ensuremath{\text{ H}_{\text{sys}}}}


%% Header
\rhead{Thad Haines \\ Page \thepage\ of \pageref{LastPage}}
\chead{Equations \\}
\lhead{Research \\ }

\begin{document}
	Relative Hz difference of PSDS - LTD $\left( \text{i.e. }  \left|f_{PSDS}(t)- f_{LTD}(t)\right| \times 60 \text{Hz} \right)$
	
	$\Delta \omega=1-\omega$
		\[ \dot{\omega}_{sys} = \dfrac{1}{2H_{sys} } \left( \dfrac{P_{acc, sys} }{\omega_{sys}(t)} - D_{sys}\Delta\omega_{sys}(t)  \right)\] 
	
	\paragraph{ACE Conventions:} Positive ACE denotes over generation. $B$ (the frequency bias) is negative.
	\begin{align*}
	\text{ACE}_{\text{tie line}} &= P_{gen} - P_{load} - P_{\text{sched interchange}}\\
	\text{ACE}_{\text{frequency bias}} &= 10B(f_{\text{actual}}-f_{\text{sched}})f_{base}\\
	\text{ACE} &= \text{ACE}_{\text{tie line}} -\text{ACE}_{\text{frequency bias}}
	\end{align*}
	
	One way to think of deviation plots is $\text{LTD}_{data}+\text{Deviation}_{data} = \text{PSDS}_{data}$. \\(Assuming all time step issues are handled appropriately.)\\

%Percent difference
\[\%_{diff} = \dfrac{\abs{x-y}}{\frac{x+y}{2}}*100\% \]
	
	Distribution of accelerating power based on inertia:
	\[ P_{e,i}(t) = P_{e,i}(t-1)-\Delta P_{acc,sys}(t)\dfrac{H_{i}}{H_{sys}} \]
	

	%%% custom commands for nicer units % in thad_cmds.tex....
	%\newcommand{\mw}{\ensuremath{\text{ MW}}}
	%\newcommand{\hz}{\ensuremath{\text{ Hz}}}
	%\newcommand{\pu}{\ensuremath{\text{ Pu}}}
	%\newcommand{\sbase}{\ensuremath{\text{S}_{\text{Base}}}}
	%\newcommand{\fbase}{\ensuremath{f_{\text{Base}}}}
	%\newcommand{\mbase}[1]{\ensuremath{\text{M}_{\text{Base}_{#1}}}}
	%\newcommand{\hsys}{\ensuremath{\text{ H}_{\text{sys}}}}
	
	The theoretical steady state frequency was calculated as
	\begin{align*}
	f_{ss} &= f_{ref}+\Delta f = f_{ref} + \frac{\Delta P}{S_{base}\beta} \numberthis 
	\end{align*}
	When $R$ is a\pu\ value, $\beta$ for $N$ governor equipped machines is calculated as
	\begin{align*}
	\beta &= \sum_{i=1}^{N} \dfrac{1}{R_i \frac{\sbase}{\mbase{i}}}\numberthis
	% from glover section 12.2 page 659
	\end{align*}
	
	Additionally, in a system with $N$ generators, the weighted system frequency, $f_{w}$, is calculated as
	\begin{align*}
	f_{w}\pu &= \sum_{i=1}^{N} \dfrac{f_i}{\fbase} \dfrac{H_{i}\mbase{i}}{\hsys} \numberthis\\
	\text{where} \hsys &= \sum_{i=1}^{N}H_i\mbase{i} \numberthis
	\end{align*}
	
	If IACE is to be added using the weight option:
	\[ACE = ACE*(1-IACEweight)+ IACE*IACEweight*IACEscale\]
	Else:
	\[ACE = ACE+IACE*IACEscale\]

 Branch Power Flow calculated as \\ % according to glover
\begin{align}
P &= \dfrac{V_R V_S}{X} \sin(\delta_S - \delta_R)\\
Q &= \dfrac{V_R}{X} \left(V_S\cos(\delta_S - \delta_R)-V_R\right)\\
S &= P + jQ\\
Amps &= \dfrac{\abs{S}}{V_S\sqrt{3}}\\
\text{where} &V_S,V_R \text{ are magnitudes.}
\end{align}



\end{document}