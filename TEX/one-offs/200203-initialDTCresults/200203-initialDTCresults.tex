\documentclass[12pt]{article}
\usepackage[english]{babel}
\usepackage[utf8]{inputenc}

%% Pointer to 'default' preamble, other reusable files
\input{../../thad_preamble.tex}
\input{../../thad_cmds.tex}

%% Header
\rhead{Thad Haines \\ Page \thepage\ of \pageref{LastPage}}
\chead{Initial Definite Time Controller Results\\ 02-03-20}
\lhead{Research \\ }

%\usepackage{graphicx}
%\graphicspath{ {figures/} }
%\newcommand{\caseName}{ }

\begin{document}
\paragraph{Scenario: } Six machine 2 hour `virtual' ramp. Loads and/or generation begin  20\% 45 minute ramp up at $t=10$ minutes, hold for 10 minutes when $t=55$ minutes, then reduce 20\% over 45 minutes.

\vspace{1em}
Definite time controllers act on bus 8 and 9 shunts to keep bus voltage within the 1.0-1.04 PU range.
Additionally, MVAR branch flow was tested as a shunt trigger.

AGC is sent to generators 1 and 3 every 30 and 45 seconds respectively.

\vspace{1em}
A third ramp was conducted without the ramping of load for a more realistic scenario.


% six machine system
\begin{figure}[!ht]
	\centering
	\footnotesize
	\includegraphics[width=\linewidth]{../../../TEX/models/sixMachine/sixMachine}
	\caption{Two Area Six Machine System.}
	\label{fig: six machine}
\end{figure}

\paragraph{Results:} 
Definite time controller appears to function as desired.
PSLTDSim could be used for capacitor switching coordination.
When load and generation ramps together odd frequency `blips' occur when shunts are switched.
When load is not ramped with generation, AGC forces generator 3 to a very low power generation point.


\newcommand{\caseName}{def}
\newcommand{\scrunch}{\vspace{-.8em}}
\pagebreak
\paragraph{CASE 1: Voltage control on shunts only. Generation and Load Ramp} \ \\
\renewcommand{\caseName}{sixMachineWindramp1}
\scrunch
\includegraphics[width=\linewidth]{figures/\caseName sysShuntV}
\scrunch
\includegraphics[width=\linewidth]{figures/\caseName MVarflow8to9}
\scrunch
\includegraphics[width=\linewidth]{figures/\caseName sysShuntMVAR}
\scrunch
\includegraphics[width=\linewidth]{figures/\caseName sysPePmLoad2}

\pagebreak
\paragraph{CASE 1: Voltage and Qbr control on shunts. Generation and Load Ramp} \ \\
\renewcommand{\caseName}{sixMachineWindramp2}
\scrunch
\includegraphics[width=\linewidth]{figures/\caseName sysShuntV}
\scrunch
\includegraphics[width=\linewidth]{figures/\caseName MVarflow8to9}
\scrunch
\includegraphics[width=\linewidth]{figures/\caseName sysShuntMVAR}
\scrunch
\includegraphics[width=\linewidth]{figures/\caseName sysPePmLoad2}

\pagebreak
\paragraph{CASE 3: Voltage control on shunts only. Generation Ramp only} \ \\
\renewcommand{\caseName}{sixMachineWindramp3}
\scrunch
\includegraphics[width=\linewidth]{figures/\caseName sysShuntV}
\scrunch
\includegraphics[width=\linewidth]{figures/\caseName MVarflow8to9}
\scrunch
\includegraphics[width=\linewidth]{figures/\caseName sysShuntMVAR}
\scrunch
\includegraphics[width=\linewidth]{figures/\caseName sysPePmLoad2}

\end{document}